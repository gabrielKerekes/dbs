\documentclass[slovak, 12pt, Times New Roman]{article}
\usepackage[slovak]{babel}
\usepackage[utf8]{inputenc} 
\usepackage[top=1in, bottom=1in, left=0.7in, right=0.7in]{geometry}
\usepackage{graphicx}
\usepackage{hyperref}
\hypersetup{
    colorlinks=true, %set true if you want colored links
    linktoc=all,     %set to all if you want both sections and subsections linked
    linkcolor=black,  %choose some color if you want links to stand out
}
\usepackage{fancyhdr}
\begin{document}
	\thispagestyle{fancy}
	\chead{Slovenská technická univerzita v Bratislave, Fakulta elektrotechniky a informatiky, Ilkovičova 3, 812 19 Bratislava}
	\topskip70mm
		\begin{center}\huge{Riaditel väznice\\\par}Semestrálna práca z DBS\end{center}
	\title{Softvérová špecifikácia WebRočenka}
	\date{}

	\begin{minipage}[b]{\textwidth}
	    \vspace{110mm}	 
	    \large   	\hspace{110mm} Vypracovali:\\
	    V Bratislave \hspace{82mm} Gabriel Kerekeš \\
	    1. Decembra 2014 \hspace{70mm} Jakub Jahnič \\
	    \vspace{-20mm} 
	\end{minipage}
	\topskip0pt

	\clearpage
	\tableofcontents
	\clearpage

	\section{Úvod}
		\subsection{Účel dokumentu}
			
		\subsection{Prehľad dokumentu}
	\section{Úloha}
		Kód úlohy: \textbf{D07} \\
		Názov úlohy: \textbf{Riaditeľ väznice} \\

		Bol Ste vymenovaný za riaditeľa menšej väznice, ktorá má A väzňov ubytovaných v B celách a máte k dispozícii C dozorcov (doplňte za A, B
		,C vhodné čísla). Vytvorte databázový systém, tak by Vaša väznica mohla efektívne fungovať. Potrebujete evidovať v databázovom systéme 
		väzňov s ich trestami a dátumom prepustenia, ich umiestnenie na celách, služby dozorcov, atď. Musíte mať prehľad o tom kedy bude ktorý 
		väzeň prepustený, sledovať jeho správanie a podľa tohto správania mu znížit, alebo zvýšiť trest. Podľa Vášho uváženia sledujte aj 
		ďalšie údaje napr. návštevy u väzňov, prijaté listy a balíčky, počet odpracovaných hodín dozorcov (prípadne aj väzňov), celkové 
		prehľady, atď.
	\section{Kocneptuálna úroveň}
		\subsection{Entity}
			Entity sme vyberali na základe toho aké údaje budeme potrebovať ukladať. Začali sme s entitami VÄZEŇ, DOZORCA, CELA.\\ Entita VÄZEŇ 
			slúži na uchovávanie informácií o väzňoch, DOZORCA na uchovávanie údajov o dozorcoch a CELA informácie o celách. Pri navrhovaní 
			entity VÄZEŇ sme boli donutení vytvoriť ďaľšie entity súvisiace s väzňami. Sú to PRÁCA, ZÁVAŽNOSŤ TRESTU, PRIESTUPOK, POŠTA. PRÁCA 
			je entita, ktorá má v sebe uložené všetky práce, ktoré väzni môžu vykonávať. ZÁVAŽNOSŤ TRESTU má v sebe uložené závažnosti trestov 
			aj s ichpopisom. PRIESTUPOK ukladá informácie o priestupkoch vykonaných väzňami. POŠTA ukladá informácie o pošte, ktorý väzni 
			prijali alebo odoslali. Rovnako k entite DOZORCA sme museli vytvoriť entity HODNOSŤ, BLOK, SLUŽBA. HODNOSŤ je tabuľka uchovávajúca 
			všetky hodnosti, ktoré môže dozorca dosiahnuť. Dozorca pracuje v určitom čase iba v jednom bloku. Všetky bloky väznice musia byť 
			niekde uložené. Preto sme vytvorili entitu BLOK, aby sme vedeli určit v akom bloku dozorca pracujem, alebo v akom bloku sa nachádza 
			cela. Entita SLUŽBA ukladá všetky služby, ktoré dozorca mal, alebo ešte len bude mať. Na určenie o akú službu sa jedná sme 
			vytvorili entitut TYP$\_$SLUŽBY aby sa typ služby určoval jednoduchšie.
		\subsection{Atribúty}
			Atribúty sú vlastnosti entít. Zvolili sme ich následovne po diskusii so zákazníkom: \\
			VÄZEŇ - id, meno, priezvisko, cela, datum prijatia, datum prepustenia, priestupok, zavaznost trestu, praca\\
			ZÁVAŽNOSŤ TRESTU - id, nazov\\
			PRIESTUPOK - id, id väzňa, dátum, popis\\
			PRÁCA - id, názov, odmena\\
			POŠTA - id, id väzňa, cinnost\\
			DOZORCA - id, meno, priezvisko, blok, hodnosť, plat \\
			HODNOSŤ - id, nazov, platová skupina\\
			PLATOVÁ SKUPINA - id, minimálny plat, maximálny plat\\
			BLOK - id, názov bloku, počet ciel, kapacita, obsadenosť, úroveň stráženia\\
			CELA - id, kapacita, obsadenosť\\
			SLUŽBA - id, id dozorcu, id typu sluzby\\
			TYP SLUŽBY - id, nazov sluzby, cas od, cas do\\
		\subsection{Matrix diagram}
			\begin{figure}[!htb]
				\centering
				\includegraphics[scale=0.45]{matrixDia.png}
				\caption{Matrix diagram}
				\label{fig:Reinforcement}
			\end{figure}
		\subsection{Vzťahy medzi entitami}
			\textbf{Väzeň - Práca} \\
				Kardinalita: N:1 Jednu prácu môže vykonávať viac väzňov, avšak väzeň môže vykonávať iba jednu prácu v daný čas.\\
				Parcialita: Väzeň nemusí vykonávať žiadnu prácu, avšak práca musí byť vykonávaná nejakým väzňom.\\ \\
			\textbf{Väzeň - Závažnosť trestu} \\
				Kardinalita: N:1 Jeden typ závažnosti trestu môže mať viac väzňov, avšak väzeň môže mať iba jeden typ závažnosti trestu. \\
				Parcialita: Väzeň musí mať nejaký typ závažnosti trestu, avšak závažnosť trestu nemusí prislúchať nejakému väzňovi.\\ \\
			\textbf{Väzeň - Priestupok} \\
				Kardinalita: 1:N Jeden priestupok bude zaznamenaný pre jedného väzňa, a jeden väzeň mohol spáchať viacej priestupkov.
				\\
				Parcialita: Väzeň nemusí spáchať priestupok, avšak priestupok musí mať svojho vinníka.\\ \\
			\textbf{Väzeň - Pošta} \\
				Kardinalita: 1:N Poštu môže odoslať/prijímať len jeden väzeň, avšak viac väzňov môže odoslať/prijímať poštu.\\
				Parcialita: Väzeň nemusí odosielať/prijímať pošty, avšak pošta musí mať svojho odosielateľa/prijímateľa.\\ \\
			\textbf{Väzeň - Cela} \\
				Kardinalita: N:1 Cela môže ubytovávať viacerých väzňov, avšak väzeň môže byť ubytovaný len v jednej cele. \\
				Parcialita: Väzeň musí byť ubytovaný v cele, avšak cela nemusí ubytovávať žiadneho väzňa.\\ \\
			\textbf{Dozorca - Hodnosť} \\
				Kardinalita: N:1 Jednu hodnosť môže mať viacero dozorcov, avšak jeden dozorca môže mať len jednu hodnosť.\\
				Parcialita: Dozorca musí mať hodnosť, a taktiež hodnosť musí prislúchať nejakému dozorcovi.\\ \\
			\textbf{Dozorca - Blok} \\ \\
				Kardinalita: N:1 Blok môže byť strážený viacerými dozorcami, avšak dozorca môže strážiť len jeden blok naraz.\\
				Parcialita: Dozorca musí strážiť nejaký blok, avšak blok nemusí byť strážený žiadnym dozorcom.\\ \\
			\textbf{Dozorca - Služba} \\
				Kardinalita: 1:N Jedna služba môže prislúchať len jednému dozorcovi, avšak jeden dozorca môže mať viacero služieb.\\
				Parcialita: Dozorca musí byť prihlásený na službu a služba musí mať svojho vykonávateľa.\\ \\
			\textbf{Hodnosť - Platová skupina} \\
				Kardinalita: N:1 Jedna platová skupina môže obsahovať viacero hodností, avšak jedna hodnosť môže byť len v jednej platovej 
				skupine.\\
				Parcialita: Hodnosť musí byť zapísaná v nejakej platovej skupine, a taktiež platová skupina musí obsahovať nejakú hodnosť.\\ \\
			\textbf{Blok - Cela} \\
				Kardinalita: 1:N Jedna cela môže byť len v jednom bloku, avšak jeden blok môže obsahovať viacero ciel.\\
				Parcialita: Blok musí obsahovať celu, a taktiež cela musí byť v nejakom bloku.\\ \\
			\textbf{Služba - Typ služby} \\
				Kardinalita: N:1 Jeden typ služby môže mať viacero služieb, avšak jedna služba môže byť len jedným typom služby.\\
				Parcialita: Služba musí byť nejakým typom služby, avšak typ služby nemusí mať žiadnu službu.\\ \\
		\subsection{ERA diagram}
			\begin{figure}[!htb]
				\centering
				\includegraphics[scale=0.45]{ERAdia.png}
				\caption{ERA diagram}
				\label{fig:Reinforcement}
			\end{figure}
		\subsection{Prehľad požiadaviek používateľa databázového systému}
			Budeme potrebovať evidovať všetkých väzňov. Ich dátumy uväznenia, dátumy prepustenia, poštu, ktorú prijímajú/odosielajú, závažnosť 
			ich trestov, ich prácu vo väznici a priestupky, ktoré vykonali. Podľa ich priestupkov budeme vedieť určovať ich správanie a podľa 
			toho im zvyšovať/znižovať ich trest. \\
			Taktiež budeme potrebovať evidovať informácie o všetkých dozorcoch. Akú majú hodnosť, plat a aký blok strážia. Každá hodnosť má 
			pridelenú platovú skupinu do ktorej zapadá. O platovej skupine potrebujeme uchovávať len jej názov, maximálny a minimálny plat. 
			Dozorcovia tiež majú služby, ktoré tiež treba uchovávať. Služby sú rôzne, preto budeme potrebovať ukladať aj typy služieb. Typ 
			služby má byť určený nejakým názvom, časom začiatku služby a časom škončenia služby. Máme tri základné typy služieb a to rannú, 
			poobednú a nočnú, ale napríklad keď dozorca môže pracovať od 10:00 do 12:00, malo by toto byť vytvorené ako nový typ služby.	\\
			Naša väznica je rozdelená na bloky. V každom bloku je niekoľko ciel a v každej cele môže bývať viacero väzňov. Čiže systém by mal 
			mať uložené všetky naše cely, ich kapacitu a v akom bloku sa nachádzajú. Pri blokoch potrebujeme mať tiež uloženú ich kapacitu, ale 
			ešte aj ich úroveň stráženia znázornenú číslom. 
	\section{Logická úroveň}
		\subsection{Normalizácia}
			Všetky naše tabuľky sú navrhnuté podľa 1., 2. aj 3. normálovej formy. Aby bola splnená prvá normálová forma, musia byť všetky 
			atribúty entít atomické, čo znamená že napr. meno a priezvisko nebudú uchované v jednom stĺpci ale budú rozdelené do dvoch stĺpcov. 
			Pre druhú normálovú formu musí byť splnené, že každý atribút, ktorý nieje primárnym kľúčom je na ňom úplne závislý. ??? V tretej 
			normálovej forme je tabuľka vtedy, keď všetky jej atribúty, ktoré nie sú primárnym kľúčom
		\subsection{Voľba a stanovenie identifikátorov, primárnych kľúčov, pre definované entity}
			Ako identifikátory môžeme zvoliť atribút, kombináciu atribútov alebo zaviesť nový atribút,
			ktorý slúži iba ako jedinečný identifikátor každého riadku. Zvolili sme práve tretiu možnosť a ku všetkým naším entitám pridáme 
			primárny kľúč ID.		
		\subsection{Relačná schéma}	
	\section{Implementačná úroveň}
		\subsection{Slovný popis návrhu}
			\subsubsection{Dátove typy}
				Mená sme zvolili ako dátový typ VARCHAR2 s rozsahom 12 a priezviská s rozsahom 18. Predpokladáme, že takýto rozsah nám bude stačiť, keďže pravdepodobne budeme pracovať so slovenskými menami.
				Pre názov práce, názov priestupku, názov hodnosti, názov služby a názov trestu sme použili dátový typ VARCHAR2 s rozsahom 12. 
				Pre úroveň stráženia a závažnosť trestu použijeme dátový typ NUMBER s rozsahom 1. Keďže úrovne stráženia a závažnosti sa značia od 1-9, tak nám takýto rozsah bude stačiť. 
				Pre plat, počet ciel, kapacitu väzňov v bloku, obsadenosť v bloku a pre pracovné odmeny sme zvolili dátový typ NUMBER s rozsahom 5. Takýto rozsah nám bude stačiť. 
				Pre každé ID sme zvolili dátový typ NUMBER s rozsahom 5 pre jasnejšiu identifikáciu a dostatočný priestor. 
				Dátumy sa ukladajú ako reťazce pohyblivej dĺžky s maximálnym rozsahom 10 - VARCHAR2(10).
				Čas od a čas po v typoch služieb ukladáme ako NUMBER(2,1), pretože sa tieto časy budú značiť ako napr. 6. alebo 6.5, či 12.5.
				Pre kapacitu väzňov a obsadenosti väzňov v celách nám stačí dátový typ NUMBER s rozsahom 1, pretože žiadna cela nebude môcť obsahovať viac ako 2 väzňov, a teda ani obsadenosť nebude vyššia. 
				Pre minimum platovej skupiny, maximum platovej skupiny a plat dozorcu budeme potrebovať dátový typ NUMBER(5,2), teda NUMBER s rozsahom 5 číslic a s presnoťou 2 desatinným miestach. Keďže plat môže byť aj napr. 540,55.
			\subsubsection{Určenie DEFAULT hodnôt pre stĺpce}
				Zadanie defaultnej hodnoty má zmysel pri vytváraní nového väzňa, kde za dátum uväznenia zadáme aktuálny dátum, získaný orezaním a zmenou typu systémového času. 
				Takisto aj dátum priestupku sa bude riešiť rovnakým spôsobom, a teda defaultnov hodnotou bude deň pripísania priestupku.
				Pre atribút práca, v entite väzeň, môžeme zadať defaultnú hodnotu NULL, pretože tento atribúť zmeníme, až keď mu dozorcovia pridelia prácu. 
				Pre obsadenosť bloku a obsadenosť ciel môžeme zadať defaultnú hodnotu NULL, pretože hneď po vytvorení v bloku nie je žiaden väzeň a takisto pri vytvorení cely nie je v cele žiaden väzeň. 
			\subsubsection{Obmedzenia}
				ja
		\subsection{Fyzický model}
		\subsection{Prehľad potrebných SQL dopytov}
			\subsubsection{Vkladanie údajov do tabuliek, zmena hodnôt stĺpcov}
				INSERT INTO VAZNI VALUES(10000,'Ferko','Mrkvcicka',default,'12/10/2022',10,default,default,5);\\
				INSERT INTO DOZORCOVA VALUES(50,'Rastislav','Dolina');\\
				INSERT INTO POSTY VALUES(11,12311,'poslal');\\
				INSERT INTO BLOKY VALUES(1432,'lave kridlo',122,244,default,3);\\
				INSERT INTO CELY VALUES(10002,2,default);\\
				INSERT INTO HODNOSTI VALUES('12','Plukovnik',1112);\\
				INSERT INTO PLATOVE$\_$SKUPINY VALUES(214,600,900);\\
				INSERT INTO PRACE VALUES(65,'pradelna',100);\\
				INSERT INTO SLUZBY VALUES(934,1235,592);\\
				INSERT INTO TYPY$\_$SLUZIEB VALUES(521,'ranna',6,14.5);\\
				INSERT INTO ZAVAZNOSTI$\_$TRESTOV VALUES(11,3);\\
				INSERT INTO PRIESTUPKY VALUES(9812,531,default,'bitka');\\
			\subsubsection{Vymazávanie údajov}
	\section{Záver}

\end{document}